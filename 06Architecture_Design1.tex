\section{Architecture Design}

\subsection{Overall Architecture}
\begin{figure}[h]
    \centering
    \includegraphics[width=0.5\textwidth]{junho/archi.png}
    \caption{Architecture}
\end{figure}

The overall system architecture consists of four tightly connected layers—
(1) data acquisition on Apple Watch,
(2) preprocessing and context integration on the iOS client,
(3) multimodal inference and personalization within the FastAPI backend, and
(4) natural-language–based appliance control via an LLM agent.
Each layer operates asynchronously but synchronizes through a unified data schema centered around the user identifier and temporal index.

\subsubsection{Sensor and Context Data Acquisition Layer (Apple Watch \& iOS)}
The acquisition layer runs on both the Apple Watch and the paired iPhone.
The Apple Watch continuously collects high-frequency physiological signals such as heart rate, heart rate variability, respiratory rate, sleep stages, and workout metrics using HealthKit and motion sensors.

The iPhone aggregates these signals with contextual information from CoreLocation (GPS traces, speed, heading) and ambient environment sources. All data points are normalized into time-series format, timestamped, and enriched with device metadata (device model, OS version, and sample source).
A local buffering module ensures reliability by temporarily storing samples when network connectivity is unavailable and flushing them to the backend when the connection is restored.

\subsubsection{Data Transmission and Preprocessing Layer (iOS → Backend)}
The client sends health and context samples to the FastAPI backend through a lightweight REST protocol.
Each batch upload includes:
\begin{itemize}
    \item user identifier and device identifier
    \item time-series physiological measurements
    \item location window and place-inference results (home/commute/office)
    \item device state snapshots (lighting, AC, humidity)
    \item weather tile index for the current GPS coordinate
\end{itemize}

Before storage, the backend applies validation (type checking, timestamp ordering, duplication removal using composite keys) and stores the data into domain-specific tables aligned with the ERD.

This layer ensures:
\begin{itemize}
    \item schema normalization,
    \item deduplication of rapidly updated sensors, and
    \item efficient time-indexed insertion through dedicated indices.
\end{itemize}

\subsubsection{Backend Inference and Context Integration Layer}

The core intelligence of the system resides in the backend, where multiple data streams are merged into a unified feature vector.

The backend periodically assembles:
\begin{itemize}
    \item last 24-hour health signal windows,
    \item matched weather observations (temperature, humidity, rainfall, PM2.5),
    \item spatial context (home/office/commuting), and
    \item user baseline statistics (HRV median, resting HR, sleep trend)
\end{itemize}

These combined features are passed through a hybrid inference module consisting of:

\begin{itemize}
    \item an LSTM encoder for temporal physiological patterns,
    \item dense layers for environmental encoding, and
    \item an attention mechanism for highlight periods that influence the user's present condition.
\end{itemize}

The model outputs four real-time indicators:
stress, fatigue, sleep readiness, and environmental comfort.

The backend maintains a preference memory that updates dynamically using:

\begin{itemize}
    \item user overrides on appliance settings,
    \item action outcome logs,
    \item RLHF-style feedback from approval/skip decisions, and
    \item historical routines extracted from GPS–time clustering.
\end{itemize}

This module ensures that recommendations become more personalized over time.

\subsubsection{LLM-Driven Appliance Control and Decision-making Layer}

Once the backend produces the user state indicators, the system composes a structured prompt and sends it to an LLM

The analyzed condition output is passed to a Generative AI module, which converts it into a structured prompt describing the user’s current physiological and environmental context. 
The AI then generates personalized appliance-control commands—for example, ``activate dehumidifying mode at 27 °C after exercise,'' or ``set lighting to 50\% brightness and 3500 K color temperature for relaxation.'' 
Each recommendation includes concise scientific rationale (e.g., HRV recovery facilitation or melatonin preservation) derived from established sources such as WHO, ISO 7730, and AASM sleep guidelines.

\subsubsection{Execution, Logging, and Closed-loop Feedback Layer}

After parsing the LLM output, the backend executes the valid commands through the appliance control interface, using direct API calls (LG ThinQ or custom device controllers).

These logs form the feedback loop that continuously refines model thresholds, user preference embeddings, and the LLM prompt template—building a personalized and adaptive automation agent.

\subsection{Directory Organization}

\begin{table}[htbp]
\caption{DIRECTORY-ORGANIZATION-IOS-APP}
\label{tab:dir_ios_app}

\begin{tabularx}{\linewidth}{|p{3.8cm}|X|}
\hline
\textbf{Directory} & \textbf{File Name} \\
\hline
\texttt{space/ (Root)} &
spaceApp.swift \newline
TimelineDataModel.swift \newline
FastAPIService.swift \newline
DeviceManager.swift \newline
FontSizeManager.swift \newline
CallHistoryManager.swift \newline
WatchConnectivityManager.swift \newline
HealthKitManager.swift \newline
SupabaseManager.swift \newline
LocationManager.swift \\
\hline
\end{tabularx}
\end{table}

\begin{table}[htbp]
\begin{tabularx}{\linewidth}{|p{3.8cm}|X|}
\hline
\texttt{space/Views/Auth} &
LoginView.swift \newline
SignUpView.swift \newline
SocialLoginButton.swift \\
\hline

\texttt{space/Views/Chat} &
CallErrorHistoryView.swift \newline
ChatView.swift \newline
PhoneCallView.swift \\
\hline

\texttt{space/Views/Common} &
ContentView.swift \newline
MainTabView.swift \newline
PersonalTestView.swift \\
\hline

\texttt{space/Views/Device} &
AddItemWidget.swift \newline
ApplianceCard.swift \newline
ApplianceItemCard.swift \newline
ApplianceView.swift \newline
DeviceCard.swift \newline
DeviceDetailView.swift \newline
DeviceView.swift \\
\hline

\texttt{space/Views/Home} &
HomeLocationSetupView.swift \newline
HomeView.swift \newline
LocationTagSheet.swift \newline
SplashView.swift \\
\hline

\texttt{space/Views/Persona} &
AdjectiveTagView.swift \newline
PersonaBubbleWidgetNew.swift \newline
PersonaChatView.swift \newline
PersonaEditView.swift \newline
PersonaListView.swift \newline
PersonaWidget.swift \\
\hline

\texttt{space/Views/Settings} &
DoNotDisturbView.swift \newline
EmergencyCallView.swift \newline
FontSizeView.swift \newline
GeneralView.swift \newline
MenuRow.swift \newline
MenuView.swift \newline
MyPageView.swift \newline
NotificationMethodView.swift \newline
SpaceNotificationView.swift \\
\hline

\texttt{space/Views/Timeline} &
StateDetailView.swift \newline
StateWidget.swift \newline
TimelineDetailView.swift \newline
TimelineWidget.swift \\
\hline

\texttt{space/Models} &
AdjectiveModels.swift \newline
TaggedLocation.swift \\
\hline

\texttt{space/Managers} &
TaggedLocationManager.swift \\
\hline

\texttt{space/Repositories} &
PersonaRepository.swift \\
\hline

\texttt{space/Services} &
SupabaseService.swift \\
\hline
\end{tabularx}
\end{table}


\begin{table}[h]
\caption{DIRECTORY-ORGANIZATION-WATCH-APP}
\label{tab:dir_watch_app}
\centering
\small
\renewcommand{\arraystretch}{1.15}
\begin{tabularx}{\linewidth}{|p{4.2cm}|X|}
\hline
\textbf{Directory} & \textbf{File Name} \\
\hline
\texttt{space Watch App/ (Root)} &
space\_Watch\_AppApp.swift \newline
ContentView.swift \newline
MapView.swift \newline
WatchConnectivityManager.swift \newline
WatchHealthKitManager.swift \newline
WatchLocationManager.swift \\
\hline

\texttt{space Watch App/Assets.xcassets} &
AppIcon.appiconset/ \newline
AccentColor.colorset/ \newline
Contents.json \\
\hline
\end{tabularx}
\end{table}

\clearpage


\begin{table}[htbp]
\caption{DIRECTORY ORGANIZATION - BACKEND}
\label{tab:dir_backend}
\centering
\small
\renewcommand{\arraystretch}{2}
\begin{tabularx}{\linewidth}{|p{3.8cm}|X|}
\hline
\textbf{Directory} & \textbf{File Name} \\
\hline
\texttt{app/ (Root)} &
\_\_init\_\_.py \newline
main.py \newline
models.py \newline
schemas.py \\
\hline

\texttt{app/api} &
\_\_init\_\_.py \newline
users.py \newline
chat.py \newline
hrv.py \newline
weather.py \newline
location.py \newline
appliance.py \newline
voice.py \newline
voice\_realtime.py \newline
characters.py \newline
sendbird\_webhook.py \newline
tracking.py \\
\hline

\texttt{app/models} &
\_\_init\_\_.py \newline
user.py \newline
hrv.py \newline
appliance.py \newline
weather.py \newline
location.py \newline
info.py \newline
tracking.py \\
\hline

\texttt{app/schemas} &
\_\_init\_\_.py \newline
user.py \newline
info.py \newline
tracking.py \\
\hline

\texttt{app/cruds} &
\_\_init\_\_.py \newline
user.py \newline
info.py \newline
tracking.py \\
\hline

\texttt{app/services} &
llm\_service.py \newline
appliance\_rule\_engine.py \newline
appliance\_control\_service.py \newline
hrv\_service.py \newline
weather\_service.py \newline
geofence\_service.py \newline
fatigue\_predictor.py \newline
sendbird\_client.py \newline
voice\_service.py \newline
realtime\_voice\_agent.py \\
\hline

\texttt{app/config} &
\_\_init\_\_.py \newline
db.py \newline
env.py \newline
sendbird.py \\
\hline

\texttt{app/migrations} &
env.py \\
\hline
\end{tabularx}
\end{table}


\begin{table}[htbp]
\caption{DIRECTORY ORGANIZATION - AI MODULE}
\label{tab:dir_ai}
\centering
\small
\renewcommand{\arraystretch}{1.15}
\begin{tabularx}{\linewidth}{|p{4.2cm}|X|}
\hline
\textbf{Directory/Component} & \textbf{Description} \\
\hline
\texttt{app/services/ llm\_service.py} &
\textbf{LLMService} class: \newline
- \_\_init\_\_() \newline
- \_build\_system\_prompt() \newline
- parse\_user\_intent() \newline
- generate\_appliance\_suggestion() \newline
- detect\_modification() \newline
- generate\_response() \newline
- generate\_geofence\_trigger() \newline
\newline
\textbf{MemoryService} class: \newline
- add\_message() \newline
- get\_history() \newline
- update\_long\_term\_memory() \newline
- get\_long\_term\_memory() \newline
\newline
\textbf{LLMAction} enum: \newline
- NONE \newline
- CALL \newline
- AUTO\_CALL \\
\hline

\texttt{app/api/chat.py} &
\textbf{RESTful Endpoints}: \newline
- POST /chat/\{user\_id\}/message \newline
- POST /chat/\{user\_id\}/approve \newline
- GET /chat/\{user\_id\}/history \newline
- DELETE /chat/\{user\_id\}/session \newline
\newline
\textbf{Request/Response Models}: \newline
- ChatMessageRequest \newline
- ChatMessageResponse \newline
- ApplianceApprovalRequest \newline
- ApplianceApprovalResponse \newline
\newline
\textbf{Session Management}: \newline
- chat\_sessions (dict) \newline
- get\_or\_create\_session() \\
\hline

\texttt{app/services/realtime \_voice\_agent.py} &
\textbf{Voice Integration}: \newline
- OpenAI Realtime API WebSocket handler \newline
- Audio streaming management \newline
- Real-time voice conversation \newline
- Geofence-triggered call support \\
\hline

\texttt{app/api/voice \_realtime.py} &
\textbf{WebSocket Endpoint}: \newline
- WS /voice/realtime/\{user\_id\} \newline
- Bidirectional audio streaming \newline
- Integration with llm\_service \\
\hline
\end{tabularx}
\end{table}