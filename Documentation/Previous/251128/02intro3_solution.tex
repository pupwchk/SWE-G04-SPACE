\subsection{Solution}
\subsubsection{Persona-Based User Understanding and Relationship Formation}
\begin{adjustwidth}{1em}{0em}
\setlength{\parindent}{1em} 
The most fundamental issue with current smart home systems is that they recognize the user merely as a command generator. To address this, we introduce an AI Persona System designed to learn from and remember all interactions with the user.

The persona continuously evolves based on the user's preferences, lifestyle patterns, conversational style, and emotional expression. When a user says, “I’m having a tough day,” the system doesn’t just process that statement. Instead, it comprehensively considers the environmental settings the user preferred in similar situations, the corresponding time of day and weather, and the user’s satisfaction afterward.

Unlike traditional systems that delete data after a set period, this approach accumulates all experiences as part of the system’s personality and knowledge.

As a result, the user gains a life partner that understands them deeply, without needing to repeat the same settings each time. This forms the core foundation for the smart home’s evolution from a simple device-control platform into a Personal Management Platform.
\end{adjustwidth}

\subsubsection{Voice-Based Contextual Understanding and Real-Time Communication}
\begin{adjustwidth}{1em}{0em}
\setlength{\parindent}{1em} 
When the system detects unusual events in the home—such as a fire alarm or the sound of a fall—it immediately calls the user to confirm the situation. During this call, it analyzes the user’s tone, speech rate, and intonation to determine whether a real emergency is occurring.

Furthermore, the system distinguishes and responds differently to the same phrase—for example, “It’s a bit cold”—depending on whether it’s spoken with a trembling voice at 3 a.m. or casually at 1 p.m.

This approach goes beyond simple text-based command processing, advancing to a level that integrates the user’s emotional state and situational context. Consequently, the system can respond appropriately to implicit expressions like “I’m feeling worn out today,” allowing users to create the desired environment through natural conversation rather than precise commands—fulfilling the role of a genuine life companion.

The system leverages LG’s existing home cameras for real-time detection and voice call functions. If a camera detects abnormal sounds—such as a fall, glass breaking, or unusual pet behavior—it stores a 10–20 second pre-buffer and calls the user to confirm the situation. The user can immediately respond during the call with “I’m okay” or “I need help,” and this feedback continuously improves the model’s accuracy.

All recorded data is automatically deleted upon user confirmation, with only the short buffer segment temporarily stored to ensure privacy protection.
\end{adjustwidth}

\subsubsection{User-Centric Environment Optimization Through Multi-Dimensional Data Integration}
\begin{adjustwidth}{1em}{0em}
\setlength{\parindent}{1em} 

A major inefficiency in current smart homes is that they control temperature, humidity, and lighting on a per-device basis—often leading to contradictions such as the air conditioner and dehumidifier running simultaneously. These systems also tend to react solely to sensor data, regardless of the user’s actual comfort level.

By integrating and analyzing biometric data and weather information, the proposed system can understand the user’s real condition and recommend appliance operations accordingly.

The model predicting the user’s condition doesn’t just process the fact that “the temperature is 28°C.” Instead, it interprets the context: “the user feels uncomfortable due to 28°C and high humidity while feeling tired.” Based on this, it comprehensively adjusts the air conditioner, dehumidifier, and lighting brightness to create an environment optimized for recovery.

The process of confirming the system’s predictions with the user enables continuous improvement, overcoming the “single-use data” problem in existing systems.

This signifies a shift from a device-centric to a user-centric paradigm. The decision-making center is no longer which appliance to turn on but what the user needs right now.

As a result, the user experiences an environment optimized for their condition without complex manual settings, while the system operates efficiently without wasting energy.
\end{adjustwidth}

\subsubsection{Predictive Lifestyle Management Through Habit Pattern Learning}
\begin{adjustwidth}{1em}{0em}
\setlength{\parindent}{1em} 

Existing smart homes rely on fragmented information—such as whether the user is currently home or not—failing to grasp the broader context or continuity of daily actions. The User Routine Tracking System analyzes travel routes, dwell times, and recurring lifestyle patterns to understand the user’s full day. It combines this with biometric and weather data to enable predictive lifestyle management.

For example, the system might determine that the user is returning home later than usual, had high travel activity, and shows signs of fatigue based on biometric data.

In response, it proactively creates an environment for recovery by dimming the lights, lowering the indoor temperature, and preparing calming music before the user arrives.

Conversely, if the system detects an unusual outing pattern on a weekend morning, it suggests environmental settings suitable for an active day.

This establishes a system that manages and cares for the user’s overall life, moving beyond simple automation. It transitions from a reactive structure (event occurs → immediate response) to a proactive one (pattern learning → situation prediction → preemptive suggestion).

Consequently, the user experiences having what they need prepared even before issuing a command, and the smart home functions as a true Personal Management Platform.

Moreover, long-term pattern learning can detect subtle changes in the user’s life rhythm, offering meaningful insights for health management and lifestyle stability.
\end{adjustwidth}