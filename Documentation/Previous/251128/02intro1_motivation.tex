\subsection{Motivation}
\subsubsection{Development of Smart Homes}
\begin{adjustwidth}{1em}{0em}
\setlength{\parindent}{1em} 
Over the past decade, smart home technology has developed rapidly. Initially, it was limited to simple voice-command-based appliance control, but with the spread of IoT technology, the level of inter-device connectivity and automation has gradually increased.

However, despite this progress, current smart home technologies still remain mechanical and function-centered. Systems simply respond and act according to user commands in a repetitive manner. At present, while “intelligent reactions based on commands” are possible, the system still lacks “emotional understanding” of those commands or awareness of the “context in which they occur.”

Entering the 2020s, modern users no longer view smart devices as mere tools but as integral parts of their lives. Smartphones, homes, and furniture are expected to exist not merely as “objects that perform functions” but as “companions that understand personal rhythms and emotions.” The true evolution of smart homes should therefore move beyond technological convenience toward expanding user experience and emotional depth. Against this backdrop, we propose the “House–Human Relationship” as a new core value of the smart home system.
\end{adjustwidth}

\subsubsection{House-Human Relationship}
\begin{adjustwidth}{1em}{0em}
\setlength{\parindent}{1em} 
In the past, the house was a physical space and a base for daily living. In modern society, however, it has transformed into more than just a place of residence — it has become a central space for psychological stability, identity, and emotional recovery. With the rise of remote work, personalized services, and generative AI, the home is increasingly evolving into an interactive interface between users and their environment.

Despite this evolution, most current smart homes still fail to consider emotional connection. Emotional states such as fatigue, happiness, loneliness, or tension are not reflected in how systems operate. The paradigm of user experience (UX) is shifting from convenience to empathy. Today’s technology focuses less on “what it can do” and more on “how well it understands me.”

When a user says “I’m tired,” the goal is not merely to respond with “You must be exhausted,” but to dim the lighting to reduce eye strain, play soft music to relieve tension, and create an emotionally responsive environment. The home should provide a sense of emotional feedback — transforming itself from a neutral space into a caring and attentive companion.

The house–human relationship thus evolves from a one-way model, where humans set and control the home, into a mutual learning and adaptive relationship in which the home observes and learns about the user. Through repeated interactions in which the home observes, remembers, and reacts to the user’s emotions and patterns — and the user, in turn, grows to trust and feel affinity toward the home — the system gradually develops its own persona. This evolution allows users to perceive their homes not as mere objects, but as beings they live together with.
\end{adjustwidth}

\subsubsection{Advancement of Generative AI and Self-Supervised Learning}
\begin{adjustwidth}{1em}{0em}
\setlength{\parindent}{1em} 
The rise of generative AI and self-supervised learning has opened the possibility for smart home systems to evolve beyond simple data collection and response into autonomously learning and evolving entities. Large language models, in particular, have reached a level where they can understand emotions, intentions, and context from human voice and dialogue, while reinforcement learning helps optimize their behavioral decision-making.

Furthermore, recent AI services are developing not merely in functionality but toward having distinct personalities. Examples such as ChatGPT, Replika, and Character\_AI demonstrate how AI systems can learn users’ conversational habits, information needs, and behavioral patterns — generating personalized responses unique to each user. These represent early examples of emotional interaction with AI, illustrating how artificial intelligence can be perceived as a relational and empathetic presence rather than a purely functional one.
\end{adjustwidth}