\subsection{Problem Statement}
\subsubsection{Limitations of Current Smart Home Systems}
\begin{adjustwidth}{1em}{0em}
\setlength{\parindent}{1em} 
Most current smart home systems are designed based on a command-driven architecture, in which device operations are primarily governed by event–trigger rules. This approach results in a structure that reacts to individual device-level inputs rather than managing the home as an integrated environment. 

Consequently, the system struggles to process multi-variable contextual conditions—for example, the combination of indoor temperature, humidity, and lighting. As a result, when the temperature is high and the humidity is elevated, the system may activate the air conditioner but fail to coordinate with the dehumidifier, leading to inefficient or excessive energy use. 

This limitation arises from the lack of a centralized hub or variable device capable of consolidating these factors into a unified decision-making framework. In essence, the actions of one device are not semantically linked to others, and the system cannot preserve or reference contextual information in subsequent interactions.

Another inherent limitation of command-based architectures lies in their inability to understand user context and emotion. Statements such as “I’m a bit tired today” do not correspond to explicit commands directed toward a particular device, and therefore elicit no system response. Moreover, identical phrases can have different meanings depending on context—saying “It’s hot” in summer differs significantly from saying the same in winter, or expressing “It’s cold” at 1 p.m. versus at 3 a.m.—yet current systems fail to distinguish these nuances. 

This issue originates from the reliance on speech-to-text conversion followed by purely textual analysis, which strips away emotional and situational cues. Additionally, incorporating multimodal contextual factors such as location, time, schedule, lifestyle patterns, recent behavior, and weather conditions requires a more sophisticated computational framework. This makes it difficult for existing systems—designed for lightweight, rule-based operation—to process emotional or contextual information while maintaining low hardware and processing overhead.
\end{adjustwidth}

\subsubsection{Information Deficiency}
\begin{adjustwidth}{1em}{0em}
\setlength{\parindent}{1em} 
Although current smart home systems have access to a vast amount of user data, they lack the ability to interpret this information contextually or utilize it continuously. Sensors and IoT appliances collect enormous volumes of environmental and behavioral data; however, such information is used primarily for real-time reactions rather than for long-term learning or personalization. 

As a result, the system must respond to each user action as if it were new, without reference to accumulated experience, making it impossible to form an adaptive or relational understanding of the user over time.

In most existing architectures, event logs and state data are retained only for a predefined short duration and are subsequently deleted. Consequently, if the system fails to learn from these data within that limited window, the information is never integrated into the AI’s internal state, and thus has no influence on its future behavior. 

Moreover, because each device typically stores its data on separate, vendor-specific servers, the system as a whole cannot consolidate relationships across devices or evolve a coherent, unified model of the household. This fragmented and short-term data management leads to a fundamental absence of memory, context, and continuity, preventing the smart home from developing a truly personalized and evolving intelligence.
\end{adjustwidth}

\subsubsection{Lack of Awareness toward Personal Management Systems}
\begin{adjustwidth}{1em}{0em}
\setlength{\parindent}{1em} 
The development of smart home technologies has so far focused primarily on the automation of device control. As a result, most smart home systems operate from a device-centered perspective rather than a human-centered one. Current platforms recognize the user merely as the issuer of control commands, not as an active participant whose physical and emotional states influence the home environment. 

Consequently, human-centered factors such as emotion, condition, daily rhythm, and behavioral patterns are rarely considered as core variables in system design. This structural limitation causes smart homes to provide only functional convenience, failing to deliver the psychological stability and lifestyle management that users increasingly expect from intelligent living spaces.

Such a limitation prevents the smart home from evolving into a Personal Management Platform—a system that not only automates tasks but also understands, supports, and manages the user’s well-being. For smart homes to mature into truly intelligent systems, they must recognize the user as more than a source of commands—as a subject of management and learning, whose behavioral and emotional data continually shape the home’s adaptive intelligence.
\end{adjustwidth}