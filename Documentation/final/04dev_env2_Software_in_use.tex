\subsection{Software in Use}
\begin{adjustwidth}{1em}{0em}
\setlength{\parindent}{1em}

\subsubsection{Apple Fitness+ / Apple Fitness (Apple)}
\begin{adjustwidth}{1em}{0em}
\setlength{\parindent}{1em}
Apple Fitness+ and its companion Apple Fitness app provide structured workout programs, personalized recommendations, and motivational tracking features based on data collected through the Apple Watch.  
The service offers real-time visualization of workout metrics—such as heart rate, calories burned, and activity ring progress—and synchronizes seamlessly across iPhone, iPad, and Apple TV.  
However, its core focus remains on promoting exercise engagement rather than integrating multi-dimensional data sources like GPS trajectories or meteorological context.  
In contrast, the proposed system expands upon this foundation by combining health, spatial, and environmental data to support adaptive and contextual behavior analysis.
\end{adjustwidth}
\vspace{1em}

\subsubsection{Fitbit-based Health Tracking Platforms}
\begin{adjustwidth}{1em}{0em}
\setlength{\parindent}{1em}
Fitbit’s health tracking ecosystem is among the most established wearable platforms, enabling users to monitor daily activity, sleep, and heart rate through connected devices and a cloud-based mobile interface.  
Its APIs and SDKs facilitate health research, allowing correlation studies on movement, sleep quality, and stress management.  
Nonetheless, Fitbit’s data model primarily concentrates on individual domains—such as physical activity or sleep—without capturing contextual environmental data.  
Our system advances this approach by integrating real-time GPS and weather data, constructing a unified schema that relates health trends with environmental and behavioral factors.
\end{adjustwidth}
\vspace{1em}

\subsubsection{AI-driven Wearable Data Recommendation Systems}
\begin{adjustwidth}{1em}{0em}
\setlength{\parindent}{1em}
Recent AI-powered fitness systems, such as the “Privacy-Preserving Personalized Fitness Recommender System (P3FitRec),” employ deep learning to analyze multi-sensor data and generate personalized exercise recommendations.  
These systems extract temporal dependencies from physiological signals like heart rate and activity duration, while preserving user privacy through federated learning and encrypted communication.  
However, their focus largely remains on recommendation accuracy rather than contextual integration.  
The present project enhances this paradigm by incorporating geolocation, environmental, and biometric data streams into a single inference pipeline, enabling the analysis of stress and wellness dynamics beyond exercise routines.
\end{adjustwidth}
\vspace{1em}

\subsubsection{Figma}
\begin{adjustwidth}{1em}{0em}
\setlength{\parindent}{1em}
\begin{figure}[h]
    \centering
    \includegraphics[width=0.1\textwidth]{junho/figma.png}
    \caption{figma}
\end{figure}
Figma is a collaborative cloud-based design platform optimized for UI/UX design, prototyping, and team collaboration.
In this project, Figma was used to design and prototype both the iPhone and Apple Watch interfaces with pixel-perfect precision.
Figma's component system and auto-layout features enabled consistent design patterns across different screen sizes, while real-time collaboration capabilities allowed seamless feedback exchange between designers and developers throughout the design process.
The combination of Figma's prototyping tools and developer handoff features improved design-to-code accuracy and reduced implementation time, allowing faster validation of user interactions and visual refinements during the development cycle.

\end{adjustwidth}

\subsubsection{Visual Studio Code}
\begin{adjustwidth}{1em}{0em}
\setlength{\parindent}{1em}
\begin{figure}[h]
    \centering
    \includegraphics[width=0.2\textwidth]{junho/vscode.jpeg}
    \caption{Visual Studio Code}
\end{figure}
Visual Studio Code is a lightweight source code editor optimized for cross-platform development, extension customization, and multi-language support. In this project, VSCode was used for editing configuration files, writing documentation in Markdown, and managing Git operations with an intuitive interface.
VSCode's extensive extension marketplace and integrated terminal enabled streamlined workflows for linting, formatting, and version control without switching contexts, while workspace settings allowed consistent coding standards across team members. The combination of VSCode's IntelliSense and multi-cursor editing improved productivity for repetitive tasks and code navigation, allowing efficient handling of non-Swift files and supporting scripts throughout the development cycle.
\end{adjustwidth}
\vspace{5em}

\subsubsection{Xcode}
\begin{adjustwidth}{1em}{0em}
\setlength{\parindent}{1em}
\begin{figure}[h]
    \centering
    \includegraphics[width=0.2\textwidth]{junho/xcode.png}
    \caption{Xcode}
\end{figure}
Xcode is Apple's integrated development environment optimized for building, testing, and debugging iOS, watchOS, and macOS applications.
In this project, Xcode was used as the primary IDE for Swift development, Interface Builder integration, and device-specific testing on iPhone and Apple Watch simulators. Xcode's Instruments profiling suite and live preview functionality enabled real-time performance monitoring and UI debugging without repeated builds, while built-in HealthKit and CoreLocation frameworks simplified integration with native sensor APIs. The combination of Xcode's code completion and refactoring tools improved development speed and code consistency, allowing rapid prototyping with immediate visual feedback throughout the development cycle.
\end{adjustwidth}
\vspace{5em}

\subsubsection{Github}
\begin{adjustwidth}{1em}{0em}
\setlength{\parindent}{1em}
\begin{figure}[h]
    \centering
    \includegraphics[width=0.1\textwidth]{junho/github.png}
    \caption{Github}
\end{figure}
GitHub is a cloud-based version control platform optimized for collaborative software development, code review, and CI/CD integration.
In this project, GitHub was used to manage source code repositories, facilitate pull request reviews, and automate testing pipelines for Swift codebases.
GitHub's branching strategy and merge conflict resolution enabled parallel feature development without disrupting the main codebase, while Actions workflows automated build verification and deployment processes for both iOS and watchOS targets.
The combination of GitHub's code review tools and issue linking improved code quality and traceability, allowing rapid iteration with confidence and maintaining a clean commit history throughout the development cycle.
\end{adjustwidth}

\subsubsection{Jira}
\begin{adjustwidth}{1em}{0em}
\setlength{\parindent}{1em}
\begin{figure}[h]
    \centering
    \includegraphics[width=0.2\textwidth]{junho/jira.png}
    \caption{Jira}
\end{figure}
Jira is a project management platform optimized for agile development, issue tracking, and sprint planning.
In this project, Jira was used to manage user stories, track bugs, and coordinate development tasks across iOS and watchOS implementations.
Jira's customizable workflow and sprint board visualization enabled clear prioritization of features and real-time progress monitoring, while integration with GitHub allowed automatic status updates based on pull request activities.
The combination of Jira's reporting dashboards and backlog management improved team velocity and accountability, allowing data-driven sprint planning and faster identification of blockers throughout the development cycle.
\end{adjustwidth}

\subsubsection{Notion}
\begin{adjustwidth}{1em}{0em}
\setlength{\parindent}{1em}
\begin{figure}[h]
    \centering
    \includegraphics[width=0.2\textwidth]{junho/notion.png}
    \caption{Notion}
\end{figure}
Notion is an all-in-one workspace platform optimized for documentation, knowledge management, and team collaboration.
In this project, Notion was used to centralize project documentation, meeting notes, and design specifications in a single accessible repository.
Notion's flexible database system and nested page structure enabled organized tracking of feature requirements and research findings, while cross-linking capabilities allowed seamless navigation between related documents and resources.
The combination of Notion's collaborative editing and commenting features improved knowledge sharing and reduced information silos, allowing team members to stay aligned on project goals and technical decisions throughout the development cycle.
\end{adjustwidth}

\subsubsection{OpenAI}
\begin{adjustwidth}{1em}{0em}
\setlength{\parindent}{1em}
\begin{figure}[h]
    \centering
    \includegraphics[width=0.2\textwidth]{junho/openai.png}
    \caption{OpenAI}
\end{figure}
OpenAI's GPT-5 Nano API is a lightweight language model optimized for efficient natural language processing with minimal latency and resource consumption. In this project, the GPT-5 Nano API was integrated to provide intelligent health insights and personalized workout recommendations based on user activity data.
The API's low-latency response time and compact model size enabled real-time text generation on-device without compromising battery life, while fine-tuning capabilities allowed customization for health and fitness domain-specific responses.
The combination of GPT-4 Nano's natural language understanding and cost-effective pricing improved user engagement with contextual suggestions, allowing scalable AI-powered features without significant infrastructure overhead throughout the deployment cycle.
\end{adjustwidth}

\subsubsection{Overleaf}
\begin{adjustwidth}{1em}{0em}
\setlength{\parindent}{1em}
\begin{figure}[h]
    \centering
    \includegraphics[width=0.2\textwidth]{junho/overleaf.png}
    \caption{Overleaf}
\end{figure}
Overleaf is a collaborative cloud-based LaTeX editor optimized for academic writing, technical documentation, and publication-ready typesetting.
In this project, Overleaf was used to author the research paper, format technical diagrams, and maintain consistent academic citation standards.
Overleaf's real-time collaborative editing and version history enabled simultaneous contributions from multiple authors without file conflicts, while rich LaTeX template library and integrated compiler provided professional formatting for figures, equations, and bibliographies.
The combination of Overleaf's track changes feature and comment system improved peer review efficiency and writing quality, allowing seamless revision cycles and faster preparation of camera-ready manuscripts throughout the publication process.
\end{adjustwidth}

