\section{App Specification}

\subsection{Splash Page Page}
\begin{figure}[ht]
    \centering
    \includegraphics[width=0.18\textwidth]{app/splash.png}
    \caption{Splash Page}
\end{figure}

\begin{adjustwidth}{1em}{0em}
\setlength{\parindent}{1em}
When the application is launched, a splash page is displayed for approximately 1--2 seconds.
This prevents a blank loading state from appearing while the application initializes essential components such as session tokens, device bindings, and background synchronization tasks.
By presenting a stable visual entry point, the splash page ensures a smooth and aesthetically consistent startup experience.
\end{adjustwidth}

% ===================== Login =====================

\subsection{Login Page}

\subsubsection{Login Interface}
\begin{figure}[H]
    \centering
    \includegraphics[width=0.18\textwidth]{app/login.png}
    \caption{Login Page}
\end{figure}

\begin{adjustwidth}{1em}{0em}
\setlength{\parindent}{1em}
Users can authenticate through multiple login methods, including email--password login, Google, Apple, Naver, and Kakao.
Email and password fields include real-time validation feedback, and incorrect formatting prevents login attempts.
A ``Forgot Password'' button directs users to Supabase's password reset workflow.
Upon successful authentication, a user session is created both in the iOS client and the backend, enabling synchronization of health data, devices, and preferences.
\end{adjustwidth}

\FloatBarrier

% ===================== Sign Up =====================

\subsection{Sign Up Page}

\subsubsection{Sign Up -- Basic Information}
\begin{figure}[H]
    \centering
    \includegraphics[width=0.18\textwidth]{app/signup.png}
    \caption{Signup Page}
\end{figure}

\begin{adjustwidth}{1em}{0em}
\setlength{\parindent}{1em}
The Sign Up page collects essential user data, including username, email address, and password.
All fields include structured validation to ensure proper formatting and compliance with minimum security requirements.
\end{adjustwidth}

\subsubsection{Sign Up -- Password Validation}
\begin{adjustwidth}{1em}{0em}
\setlength{\parindent}{1em}
Password requirements include a minimum length and a combination of character types.
Visual indicators provide real-time feedback, turning green when conditions are satisfied and red when they are not.
The ``Sign Up'' button becomes active only when all constraints are fulfilled.
\end{adjustwidth}

\subsubsection{Sign Up -- Consent Agreement}
\begin{adjustwidth}{1em}{0em}
\setlength{\parindent}{1em}
Users must acknowledge the Terms of Service and Privacy Policy.
The system prevents account creation unless all required consents are explicitly confirmed.
\end{adjustwidth}

\subsubsection{Sign Up -- Registration Completion}
\begin{adjustwidth}{1em}{0em}
\setlength{\parindent}{1em}
Once validated, the system creates a Supabase user record and initializes preset configurations on the backend.
These include default appliance preferences, persona profiles, and personalized environment parameters essential for downstream inference tasks.
\end{adjustwidth}

\FloatBarrier
% --------------------------------------------------

\subsection{Home Page}

\subsubsection{My Home Page}
\begin{figure}[ht]
    \centering
    % 첫 번째 홈 화면
    \begin{minipage}[b]{0.22\textwidth}
        \centering
        \includegraphics[width=\linewidth]{app/home_none.png}
        \caption{Home Page1}
    \end{minipage}
    \hspace{0.02\textwidth}
    % 두 번째 홈 화면
    \begin{minipage}[b]{0.22\textwidth}
        \centering
        \includegraphics[width=\linewidth]{app/home_use.png}
        \caption{Home Page2}
    \end{minipage}
\end{figure}
\begin{adjustwidth}{1em}{0em}
\setlength{\parindent}{1em}
The Home page provides an overview of the user's health and contextual data.
This includes the timeline widget for movement summaries, HRV-based wellness indicators, and the currently active AI personas.
\end{adjustwidth}

\subsubsection{Connected Devices}
\begin{adjustwidth}{1em}{0em}
\setlength{\parindent}{1em}
Paired devices such as Apple Watch and iPhone are displayed with real-time connection status, battery levels, and last synchronization timestamps.
\end{adjustwidth}

\subsubsection{Home Appliances Overview}
\begin{adjustwidth}{1em}{0em}
\setlength{\parindent}{1em}
All registered appliances are displayed as horizontally scrollable cards showing key parameters such as power state, temperature, and humidity.
Each card includes quick toggles and navigational access to detailed controls.
\end{adjustwidth}

% --------------------------------------------------

\subsection{Appliances Page}
\subsubsection{Appliance Home Page}
\begin{figure}[ht]
    \centering
    \includegraphics[width=0.18\textwidth]{app/appliances_home.png}
    \caption{Appliance Home Page}
\end{figure}

\begin{adjustwidth}{1em}{0em}
\setlength{\parindent}{1em}
The appliance home page displays all devices registered under the user’s account.  
Devices are automatically grouped by type—air conditioner, lighting, air purifier, humidifier, dehumidifier, and television.  
Each appliance is represented as a card component showing:

\begin{itemize}
    \item \textbf{Current Status:} ON/OFF indicator with color-coded visual feedback.
    \item \textbf{Primary Parameter:} e.g., temperature, brightness, or humidity.
    \item \textbf{Room Location:} device placement such as “Living Room” or “Bedroom.”
    \item \textbf{Operation Mode:} e.g., Auto, Sleep, Turbo, or Manual.
\end{itemize}

Users can perform quick toggles directly on each card for immediate control, while the interface applies optimistic UI updates to maintain responsiveness.  
A pull-to-refresh action retrieves the latest state from the backend, and tapping on any appliance card opens the detailed control interface.
\end{adjustwidth}

% --------------------------------------------------
\subsubsection{Appliance Detail Page}
\begin{figure}[ht]
    \centering
    \includegraphics[width=0.18\textwidth]{app/appliance_detail.png}
    \caption{Appliance Detail Page}
\end{figure}

\begin{adjustwidth}{1em}{0em}
\setlength{\parindent}{1em}
Each appliance type has a dedicated detailed control page that exposes adjustable operational parameters.  
All settings are synchronized in real time with the backend FastAPI controller.  
Supported device types and available control functions include:

\begin{itemize}
    \item \textbf{Air Conditioner:} mode (Cooling, Heating, Auto), temperature (16–30°C), and fan speed levels (1–5).
    \item \textbf{Lighting:} brightness (0–100\%), color temperature (2700K–6500K), and scene presets (Reading, Relax, Night).
    \item \textbf{Air Purifier:} fan speed control, air quality indicator (PM10, PM2.5), and mode selection (Auto, Turbo, Sleep).
    \item \textbf{Humidifier / Dehumidifier:} humidity target setting and mist/fan intensity.
    \item \textbf{TV:} volume, input source, and page brightness.
\end{itemize}

Each change triggers an API request to update device states on the backend, and a confirmation response is reflected immediately in the frontend through state binding.  
In case of communication failure, the system rolls back to the previous state to ensure interface consistency and prevent user confusion.
\end{adjustwidth}

% --------------------------------------------------
\subsection{Chat Page}

\subsubsection{General User Conversation}
\begin{figure}[ht]
    \centering
    \includegraphics[width=0.18\textwidth]{app/General_User_Conversatiol.png}
    \caption{General User Conversatiol Page}
\end{figure}
\begin{adjustwidth}{1em}{0em}
\setlength{\parindent}{1em}
The Chat page lists all AI personas registered by the user.
Each persona is displayed with a nickname, adjective tags describing its personality, and the time of the last interaction.
When the user selects a persona, a real-time conversation window opens, powered by Sendbird and an LLM-based dialogue engine.
In this mode, the user can ask general questions, receive lifestyle or wellness guidance, and engage in free-form conversation without necessarily controlling any devices.
\end{adjustwidth}

\subsubsection{Appliance Control via user}
\begin{figure}[ht]
    \centering
    \includegraphics[width=0.18\textwidth]{app/Appliance_Control_Via_User_Chat_Page.png}
    \caption{Appliance Control Via User Chat Page}
\end{figure}
\begin{adjustwidth}{1em}{0em}
\setlength{\parindent}{1em}
Beyond casual conversation, the chat interface allows the user to control home appliances using natural language commands.
For example, the user may request, ``Make the living room cooler'' or ``Turn off the bedroom lights at 11 p.m.''
The LLM interprets the intent, maps it to specific appliance actions (device type, location, target value), and generates a structured control request.
This request is transmitted to the backend smart-control API, which updates the corresponding appliance states.
Confirmation messages are then returned to the chat window so that the user can verify which devices were changed and how.
\end{adjustwidth}

\subsubsection{Trigger-Based Notifications and Suggested Control}
\begin{figure}[ht]
    \centering
    % 트리거 알림 화면
    \begin{minipage}[b]{0.22\textwidth}
        \centering
        \includegraphics[width=\linewidth]{app/Trigger-Based_Notifications.png}
        \caption{Trigger-Based Notification}
    \end{minipage}
    \hspace{0.02\textwidth}
    % 추천 컨트롤 화면
    \begin{minipage}[b]{0.22\textwidth}
        \centering
        \includegraphics[width=\linewidth]{app/Suggested_Control.png}
        \caption{Suggested Control Chat Page}
    \end{minipage}
\end{figure}

% --------------------------------------------------

\subsection{Timeline Page}

\begin{figure}[ht]
    \centering
    \includegraphics[width=0.18\textwidth]{app/timeline.png}
    \caption{Timeline Page}
\end{figure}

\begin{adjustwidth}{1em}{0em}
\setlength{\parindent}{1em}
The Timeline page visualizes the user's daily movement combined with physiological metrics.
Its main features include:
\begin{itemize}
    \item GPS polylines representing movement routes,
    \item Checkpoints with heart rate, HRV, steps, and calorie data,
    \item Weather annotations such as temperature, humidity, and particulate matter levels,
    \item Mini-map previews for past records, and
    \item Start/Stop Recording controls.
\end{itemize}
This page integrates Apple Watch health data with CoreLocation routes for a comprehensive behavioral overview.
\end{adjustwidth}

% --------------------------------------------------

\subsection{Persona Page}

\begin{figure}[ht]
    \centering
    \includegraphics[width=0.18\textwidth]{app/persona.png}
    \caption{Persona Page}
\end{figure}

\begin{adjustwidth}{1em}{0em}
\setlength{\parindent}{1em}
The Persona page allows the creation and management of AI personas.
Users can:
\begin{itemize}
    \item Add new personas by selecting nickname and character traits,
    \item Activate up to five personas simultaneously,
    \item Edit or delete existing personas, and
    \item Use personas as conversational agents for controlling appliances.
\end{itemize}
\end{adjustwidth}

% --------------------------------------------------

\subsection{Menu Page}

\begin{figure}[ht]
    \centering
    \includegraphics[width=0.18\textwidth]{app/menu.png}
    \caption{Menu Page}
\end{figure}

\begin{adjustwidth}{1em}{0em}
\setlength{\parindent}{1em}
The Menu page serves as the settings and profile management hub.
It includes:
\begin{itemize}
    \item User profile details (name, email),
    \item Account settings and general preferences,
    \item Font size settings,
    \item Notification configuration,
    \item Do-Not-Disturb scheduling,
    \item Auto-tracking toggle, and
    \item Logout button.
\end{itemize}
Logging out clears all client-side session data and disconnects active messaging channels.
\end{adjustwidth}

% --------------------------------------------------

\subsection{Settings Page}

\begin{figure}[ht]
    \centering
    \includegraphics[width=0.18\textwidth]{app/settings.png}
    \caption{Settings Page}
\end{figure}

\begin{adjustwidth}{1em}{0em}
\setlength{\parindent}{1em}
The Settings page provides global configuration options such as:
\begin{itemize}
    \item Font scaling options,
    \item Notification settings,
    \item Do-Not-Disturb settings, and
    \item Auto-tracking permissions.
\end{itemize}
User preferences persist across sessions using HARU's settings manager.
\end{adjustwidth}